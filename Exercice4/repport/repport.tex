\documentclass[a4paper,11pt]{article}
\usepackage[utf8]{inputenc}
\usepackage[T1]{fontenc}
\usepackage[english]{babel}
\usepackage{amsmath,amssymb,amsthm,amsfonts}
\usepackage[demo]{graphicx} % REMOVE "demo" to include figures!

\title{
	Computer Exercise 4\\
	EL2520 Control Theory and Practice
}
\author{
	Jean-Alix David\\
	jadavid@kth.se\\
	900612T314
	\and
	Kilian Demeulemeester\\
	kiliande@kth.se\\
	910308T213
}

\newcommand{\image}[3][width=1.0\columnwidth]{
	\begin{figure}[h!]
		\centering
	    \includegraphics[#1]{#2}
		\caption{#3}
		\label{fig:#2}
	\end{figure}
}

\begin{document}
	\maketitle

	% Minimum phase case
	\section*{Minimum phase case}

	\subsection*{Dynamic decoupling}
	The dynamic decoupling in exercise 3.2.1 is
	\[
        W(s) = \left[\begin{array}{cc} 
            1 & \frac{-.01476}{s + .0213} \\
    \frac{-.01336}{s + .02572} & 1 \end{array} \right]
	\]

	\image{figure_1.pdf}{Bode diagram of $\tilde{G}(s)$ derived in exercise 3.2.1}
	\image{figure_2.pdf}{Simulink plots from exercise 3.2.4}

	Is the controller good?
	\par\dotfill\par\dotfill\par
	Are the output signals coupled?
	\par\dotfill\par\dotfill

	\subsection*{Glover-MacFarlane robust loop-shaping}

	\image{figure_3.pdf}{Simulink plots from exercise 3.3.4}

	What are the similarities and differences compared to the nominal design?
	\par\dotfill\par\dotfill

	% Non-minimum phase case
	\section*{Non-minimum phase case}

	\subsection*{Dynamic decoupling}
	The dynamic decoupling in exercise 3.2.1 is
	\[
		W(s) = \ldots
	\]

	\image{figure_4.pdf}{Bode diagram of $\tilde{G}(s)$ derived in exercise 3.2.1}
	\image{figure_5.pdf}{Simulink plots from exercise 3.2.4}

	Is the controller good?
	\par\dotfill\par\dotfill\par
	Are the output signals coupled?
	\par\dotfill\par\dotfill

	\subsection*{Glover-MacFarlane robust loop-shaping}

	\image{figure_6.pdf}{Simulink plots from exercise 3.3.4}

	What are the similarities and differences compared to the nominal design?
	\par\dotfill\par\dotfill
\end{document}
