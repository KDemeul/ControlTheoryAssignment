\documentclass[a4paper,11pt]{article}
\usepackage[utf8]{inputenc}
\usepackage[T1]{fontenc}
\usepackage[english]{babel}
\usepackage{amsmath,amssymb,amsthm,amsfonts}
\usepackage[demo]{graphicx} % REMOVE "demo" to include figures!

\title{
	Computer Exercise 3\\
	EL2520 Control Theory and Practice
}
\author{
	Jean-Alix David\\
	jadavid@kth.se\\
	90061T314
	\and
	Kilian Demeulemeester\\
	kiliande@kth.se\\
	910308T213
}

\newcommand{\image}[3][width=1.0\columnwidth]{
	\begin{figure}[h!]
		\centering
	    \includegraphics[#1]{#2}
		\caption{#3}
		\label{fig:#2}
	\end{figure}
}

\begin{document}
	\maketitle

	% Suppression of disturbances
	\section*{Suppression of disturbances}

	The weight is
	\begin{align*}
		W_S(s) &= \ldots
	\end{align*}

	\image{figure_1.pdf}{Simulation results}

	What amplification is required for a P-controller to get the same performance, and what are the disadvantages of such a controller?
	\par\dotfill\par\dotfill\par


%	\image{figure_1.pdf}{Bode diagram of $\tilde{G}(s)$ derived in exercise 3.2.1}
%	\image{figure_2.pdf}{Simulink plots from exercise 3.2.4}
%
%	Is the controller good?
%	\par\dotfill\par\dotfill\par

	% Robustness
	\section*{Robustness}
	What is the condition on $T$ to guarantee stability according to the small gain theorem?
	\par\dotfill\par\dotfill\par
	The weights are
	\begin{align*}
		W_S(s) &= \ldots\\
		W_T(s) &= \ldots
	\end{align*}

	\image{figure_2.pdf}{Simulation results}

	Compare the results to the previous simulation
	\par\dotfill\par\dotfill\par

	% Control signal
	\section*{Control signal}

	The weights are
	\begin{align*}
		W_S(s) &= \ldots\\
		W_T(s) &= \ldots\\
		W_U(s) &= \ldots
	\end{align*}

	\image{figure_3.pdf}{Simulation results}

	Compare the results to the previous simulations
	\par\dotfill\par\dotfill\par


\end{document}
