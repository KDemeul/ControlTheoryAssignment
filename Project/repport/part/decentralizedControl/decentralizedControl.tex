In the following section we will test our decentralized controllers and analyse the impact of externals perturbations (pouring a cup of water in one of the lower tank and opening an extra outlet in one of the upper tanks).

Our controller are used in small deviation cases ($\approx 5\%$) around the equilibrium point (set as $50\%$ for each pump). 

For Figures \ref{min_dyn_fig} to \ref{nonmin_glover_fig}, each of the following experiment is done:
\begin{shortitemize}
    \item Step response relative to input 1,
    \item Cup of water in tank 1,
    \item Extra outlet in tank 3.
\end{shortitemize}

The time to reject a perturbation is measured as the time needed by the system to return to a value $\pm 1$ of the set point.

\begin{table}[h!t]
    \centering
    \footnotesize
    \begin{tabular}{c|cccc}
        & Rise & \multirow{2}*{Overshoot} & \multicolumn{2}{c}{Disturbance} \\
        & time & & Cup & Outlet\\
        \\
        & \multicolumn{4}{c}{\emph{Decentralized control}} \\
        Minimum phase & 44s & 22\% & 58s & 0s \\
        Non-minimum phase & 228s & 24\% & 221s & 172s\\
        \\
        & \multicolumn{4}{c}{\emph{Glover-MacFarlane control}} \\
        Minimum phase & 19s & 2.2\% & 36s  & 44s \\
        Non-minimum phase & 84s & 4\% & 111s & 221s \\
    \end{tabular}
    \caption{Step response and load disturbance analysis}
    \label{analysis}
\end{table}
\subsection{Decentralized control}

\subsubsection{Exercise}

We want to design a lead-lag controller which eliminates the static control error for a step response in the reference signal.

The controller transfer function is the following:

$$ F(s) = K \frac{\tau_D s + 1}{\beta \tau_D s + 1} \frac{\tau_I s + 1}{\tau_I s + \gamma}$$

We want to fulfill the following criteria:
\begin{itemize}
 \item Phase margin of $30^{\circ}$ at the cross-over frequency $\omega_c = 0.4$ rad/s.
 \item No static control error for a step response 
\end{itemize}

\subsubsection{Exercise} 

The main differences between the minimum phase case and the non-minimum phase case using a decentralized controller are:
\begin{shortitemize}
    \item The non-minimum phase system is much slower (both for step response and disturbance rejection);
    \item The step response of the non-minimum phase system starts by decreasing its value (wrong direction!);
    \item Disturbances on the top tanks have more impact on the non-minimum phase system (whereas the perturbation on the lower tanks have a similar impact on both systems).
\end{shortitemize}


